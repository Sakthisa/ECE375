% template created by: Russell Haering. arr. Joseph Crop
\documentclass[12pt,letterpaper]{article}
\usepackage{anysize}
\marginsize{2cm}{2cm}{1cm}{1cm}

\begin{document}

\begin{titlepage}
    \vspace*{4cm}
    \begin{flushright}
    {\huge
        ECE 375 Homework 1\\[1cm]
    }
    \end{flushright}
    \begin{flushleft}
    Lab Time: Wednesday 5-7
    \end{flushleft}
    \begin{flushright}
    Jacques Uber
    \vfill
    \end{flushright}

\end{titlepage}

\section{Questions}
\begin{enumerate}
    \item
    Consider the pseudo-CPU discussed in class. Suppose the instruction format is 16 bits and is
    divided into three fields: opcode field, addressing mode (AM) field that specifies direct,
    indirect, indirect with pre-decrement, and indirect with post-increment addressing mode, and an
    address field as shown below

    \begin{enumerate}
        \item
        How many bits are in the opcode field, the addressing mode field, and
        the address field? What is the maximum number of opcodes that can be
        incorporated into the CPU? What is the size of the memory in term of number of
        16-bit words?

        Text of the first part of the answer

        \item
        How many bits are in the registers PC, MAR, MDR, IR, and AC?

        Text of the second part of the answer
    \end{enumerate}


    \item
    Consider the following hypothetical 1-address assembly instruction called “Add
    Then Store Indirect with Post- increment” of the form
    \begin{verbatim}
         ADDST (x)+         ; M(M(x)) ← AC + M(M(x)), M(x) ← M(x)+1
    \end{verbatim}
    Note that this instruction does not modify the original content of the AC.
    Suppose we want to implement this instruction on the pseudo-CPU discussed in
    class. An instruction consists of 16 bits: A 4-bit opcode (includes I- bit) and
    a 12-bit address. All operands are 16 bits. PC and MAR each contain 12 bits.
    AC, MDR, and TEMP each contain 16 bits, and IR is 4 bits. Give the sequence of
    microoperations required to implement the Execute cycle (Fetch cycle is given
    below) for the above ADDST (x)+ instruction. Your solution should result in
    exactly 10 microoperations. Assume PC is currently pointing to the ADDST
    instruction and only PC and AC have the capability to increment/decrement
    itself.
    
    Answer

    \item
    Consider the structure of the simple-CPU discussed in class augmented with a
    single-port register file (i.e., only one register value can be read at a time)
    containing 32 8-bit registers (R31-R30). Suppose the pseudo-CPU can be used to
    implement the AVR instruction ST Y, R5. Give the sequence of microoperations
    required to Fetch and Execute AVR’s ST Y, R5 instruction. Your solutions should
    result in exactly 6 cycles for the fetch cycle and 4 cycles for the execute
    cycle. You may assume only the AC and PC registers have the capability to
    increment/decrement itself. Assume MDR register is 8-bit wide (which implies
    that memory is organized into consecutive addressable bytes), and AC, PC, IR,
    and MAR are 16-bit wide. Also, assume Internal Data Bus is 16-bit wide and can
    handle 8-bit or 16-bit (as well as portion of 8-bit or 16-bit) transfers in one
    microoperation.  Note that if you transfer an 8-bit value to either upper or
    lower bytes of a register REG, you must specify it as REG(H) or REG(L),
    respectively.


    Answer

    \item
    Based on the initial register and data memory contents shown below
    (represented in hexadecimal), show how these contents are modified (in
    hexadecimal) after executing each of the following AVR assembly instructions.
    Do not be concerned about what happens to the Status Register (SREG) after the
    operation. Instructions are unrelated.

    \begin{enumerate}
        \item
        Text of the first part of the answer

        \item
        Text of the second part of the answer

        \item
        Text of the first part of the answer

        \item
        Text of the second part of the answer

        \item
        Text of the second part of the answer
    \end{enumerate}

\end{enumerate}

\end{document}
