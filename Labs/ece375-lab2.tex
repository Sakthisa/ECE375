% template created by: Russell Haering. arr. Joseph Crop
\documentclass[12pt,letterpaper]{article}
\usepackage{anysize}
\marginsize{2cm}{2cm}{1cm}{1cm}

\begin{document}

\begin{titlepage}
    \vspace*{4cm}
    \begin{flushright}
    {\huge
        ECE 375 Lab 2\\[1cm]
    }
    {\large
        C - Assembly - Machine Code - TekBot
    }
    \end{flushright}
    \begin{flushleft}
    Lab Time: Wednesday 5-7
    \end{flushleft}
    \begin{flushright}
    Jacques Uber
    \vfill
    \rule{5in}{.5mm}\\
    TA Signature
    \end{flushright}

\end{titlepage}

\section{Additional Questions}
\begin{enumerate}
    \item
    This lab required you to begin using new tools for compiling and downloading code to your
    AVR-enabled TekBot using the C language. Explain why it is beneficial to write code in a
    language that can be ‘cross compiled.’ Also explain some of the problems of writing in this way.

    The benifit is that if I write code that needs to run on different different chips for slightly
    different ISA's, having the code in a high level language can abstract some of the details of
    how my code will run. If I were to write assembly code for the atmega128 it could not be ported
    to another ISA as easily as code written in C and then compiled for atmega128 could have been.
    \item
    Your program does essentially the same thing as the assembly program you downloaded for Lab 1.
    Compare the size of the output hex files and explain the differences in size between them.

    \begin{enumerate}
        \item
        The two hex files are different. Machine code generated by a computer is going to be different
        than code written by a human. Machine code generated by one assembler is going to be different
        than the code generated by another assembler.
        \item
        The size of the two programs was different.
\begin{verbatim}
ls -lah Lab1/ece375-L1_v2.hex Lab2/ece375-L2_skeleton.hex 
-rw-rw-r--. 1 uberj uberj 485 Jan 20 18:42 Lab1/ece375-L1_v2.hex
-rw-rw-r--. 1 uberj uberj  14K Jan 18 17:43 Lab2/ece375-L2_skeleton.hex
\end{verbatim}
        The assembly code compiled to a size of 485 bytes. The C code compiled to a size of 14K
        bytes. The assembly was probably written much more efficiently and had less setup code than
        machine code generated by the compiler.

    \end{enumerate}
\end{enumerate}


\section{Source Code}
\begin{verbatim}
#define F_CPU 16000000
#include <avr/io.h>
#include <util/delay.h>
#include <stdio.h>

int main(void)
{

    DDRB =0b11110000;
    PORTB=0b11110000;

    while (1) {
        PORTB=0b01100000;
        // If we get hit on left
        if ( !(PIND & 1) ){
            // Go back
            PORTB=0b00000000;   //Reverse
            _delay_ms(500);
            // turn right
            PORTB=0b00100000;   //Turn left
            _delay_ms(1000);
        }
        // If we get hit on right
        else if ( !(PIND & 2) ) {
            // Go back
            PORTB=0b00000000;   //Reverse
            _delay_ms(500);
            // turn right
            PORTB=0b01000000;   //Turn Right
            _delay_ms(1000);
        }
        // If both are touched
        else if ( !(PIND & 3) ){
            // Go back
            PORTB=0b00000000;   //Reverse
            _delay_ms(500);
            // turn right
            PORTB=0b00100000;   //Turn Left
            _delay_ms(1000);
        }
    };
}
\end{verbatim}
\section{Challenge Code}

\begin{verbatim}
#define F_CPU 16000000
#include <avr/io.h>
#include <util/delay.h>
#include <stdio.h>

int main(void)
{

    DDRB =0b11110000;   //Setup Port B for Input/Output
    PORTB=0b11110000;   //Turn off both motors

    while (1) { // Loop Forever
        // Your code goes here
        PORTB=0b01100000; //Make TekBot move forward
        // right
        if ( !(PIND & 1) ){
            // Go forward
            PORTB=0b01100000;//forward
            _delay_ms(1000);

            // go back
            PORTB=0b00000000;
            _delay_ms(500);
            // turn right
            PORTB=0b00100000;//Turn right
            _delay_ms(500);
        }
        // If we get hit on left
        else if ( !(PIND & 2) ) {

            // Go forward
            PORTB=0b01100000;//forward
            _delay_ms(1000);

            // Go back
            PORTB=0b00000000;//Reverse
            _delay_ms(500);

            // turn left
            PORTB=0b01000000;//Turn left
            _delay_ms(500);
        }
        else if ( !(PIND & 3) ){
            // Go forward
            PORTB=0b01100000;//forward
            _delay_ms(1000);

            // Go back
            PORTB=0b00000000;//Reverse
            _delay_ms(500);
            // turn left
            PORTB=0b01000000;//Turn left
            _delay_ms(500);
        }
    };
}
\end{verbatim}
\end{document}
