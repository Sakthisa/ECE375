% template created by: Russell Haering. arr. Joseph Crop
\documentclass[12pt]{article}
\usepackage[hmargin=1in, vmargin=1in]{geometry}
\usepackage{fancyhdr}
\pagestyle{fancy}
\usepackage[hang,small]{caption}
\usepackage{lastpage}
\usepackage{graphicx}
\usepackage{verbatim}
\DeclareGraphicsExtensions{.jpg}
\usepackage{url}

\def\author{Jacques Uber}
\def\title{Lab4 Pre-Lab Questions}
\def\date{\today}

\fancyhf{} % clear all header and footer fields
\fancyhead[LO]{\author}
\fancyhead[RO]{\date}
% The weird spacing here is to get the spacing of \thepage to be right.
\fancyfoot[C]{\thepage\
                    / 7}

                    \setcounter{secnumdepth}{0}
                    \setlength{\parindent}{0pt}
                    \setlength{\parskip}{4mm}
                    \linespread{1.4}


\begin{document}
\fancyhf{} % clear all header and footer fields
\fancyhead[LO]{\author}
\fancyhead[RO]{\date}
\fancyhead[CO]{\title}



\section{Pre-Lab}
\begin{enumerate}
    \item
    For this lab, you will be asked to perform arithmetic operations on numbers that are larger than
    8-bits. To do this, you should understand the different arithmetic operations supported by the AVR
    Architecture. List and describe all the different forms of ADD, SUB, and MUL (i.e. ADC, SUBI, MULF,
    etc.).

    \begin{verbatim}
    ADC: Add with carry.

    ADD: Add without carry.

    ADIW: Add an immediate value (0 - 63) to a register.

    SUB: Subtract and immediate value from a register.

    SBC: Subtract two registers using the C flag.

    SBCI: Subtracts an immediate value using the C flag from a register.

    SBIW: Subtract and immediate from a register and places the result into the register.

    MUL: Multiplies two unsigned 8-bit numbers.

    MULS: Multiplies two signed 8-bit numbers.

    MULSU: Multiple two 8-bit numbers. One signed and one unsigned.

    FMUL: Fractional multiply unsigned. Multiplies two 8-bit numbers and multiplies the result by two.

    FMULS: Fractional multiply but with signed numbers.
    \end{verbatim}

    \item
    Write pseudo-code that describes a function that will take two 16-bit numbers in data memory
    addresses \$0110-\$0111 and \$0121- \$0122 and add them together. The function will then store the
    resulting 16-bit number at the address \$0100-\$0101. (Hint: The upper address corresponds to the high
    byte of the number and don't forget about the carry in bit.)

    \begin{verbatim}
    r0 = X ; X = 0110
    r1 = Y ; Y = 0121
    ADC r0, r1
    LOW(Z) = r0
    HIGH(Z) = C ; carry bit
    \end{verbatim}

\end{enumerate}
\end{document}
