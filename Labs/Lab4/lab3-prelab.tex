% template created by: Russell Haering. arr. Joseph Crop
\documentclass[12pt]{article}
\usepackage[hmargin=1in, vmargin=1in]{geometry}
\usepackage{fancyhdr}
\pagestyle{fancy}
\usepackage[hang,small]{caption}
\usepackage{lastpage}
\usepackage{graphicx}
\usepackage{verbatim}
\DeclareGraphicsExtensions{.jpg}
\usepackage{url}

\def\author{Jacques Uber}
\def\title{Lab3 Pre-Lab Questions}
\def\date{\today}

\fancyhf{} % clear all header and footer fields
\fancyhead[LO]{\author}
\fancyhead[RO]{\date}
% The weird spacing here is to get the spacing of \thepage to be right.
\fancyfoot[C]{\thepage\
                    / 7}

                    \setcounter{secnumdepth}{0}
                    \setlength{\parindent}{0pt}
                    \setlength{\parskip}{4mm}
                    \linespread{1.4}


\begin{document}
\fancyhf{} % clear all header and footer fields
\fancyhead[LO]{\author}
\fancyhead[RO]{\date}
\fancyhead[CO]{\title}



\section{Pre-Lab}
\begin{enumerate}
    \item
    What is the stack pointer? How is the stack pointer used and how do you initialize it? Provide
    pseudo-code (NOT actual assembly code) that illustrates how to initialize the stack pointer.

    The stack pointer points to the top of the stack. This is how I would initialize it.
    \begin{verbatim}
    STACK_BASE = 0x00B0
    Move STACK_BASE into SP
    \end{verbatim}


    \item
    What does the AVR instruction LPM do and how do you use it? Provide pseudo-code that shows how
    to use the LPM instruction.

    It loads one byte from program memory into a register.
    \begin{verbatim}
    X = 0xDEAD
    LPM R3, X
    \end{verbatim}

    \item
    Look at the definition file m128def.inc. What is contained within this definition file? What are
    the benefits of using this definition file?  How would you include it into your AVR Assembly
    program?

    The m128def.inc file defines a lot of usefull constants (like PIN values). You include it by
    haveing the following line in your .asm file.
    \begin{verbatim}
    .include "m128def.inc"
    \end{verbatim}

\end{enumerate}
\end{document}
