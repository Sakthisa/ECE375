% template created by: Russell Haering. arr. Joseph Crop
\documentclass[12pt]{article}
\usepackage[hmargin=1in, vmargin=1in]{geometry}
\usepackage{fancyhdr}
\pagestyle{fancy}
\usepackage[hang,small]{caption}
\usepackage{lastpage}
\usepackage{graphicx}
\usepackage{verbatim}
\DeclareGraphicsExtensions{.jpg}
\usepackage{url}

\def\author{Jacques Uber}
\def\title{Lab5 Pre-Lab Questions}
\def\date{\today}

\fancyhf{} % clear all header and footer fields
\fancyhead[LO]{\author}
\fancyhead[RO]{\date}
% The weird spacing here is to get the spacing of \thepage to be right.
\fancyfoot[C]{\thepage\
                    / 7}

                    \setcounter{secnumdepth}{0}
                    \setlength{\parindent}{0pt}
                    \setlength{\parskip}{4mm}
                    \linespread{1.4}


\begin{document}
\fancyhf{} % clear all header and footer fields
\fancyhead[LO]{\author}
\fancyhead[RO]{\date}
\fancyhead[CO]{\title}



\section{Pre-Lab}
\begin{enumerate}
    \item
    In computing there are traditionally two ways for a microprocessor to listen to other devices and
    communicate. These two methods are commonly called ‘polling’ and ‘interrupts.’ A large amount of
    information about these two methods exists. Please describe what each of them is and a few examples
    where you would choose one over the other.

    Polling is used when you are handling events that are synchronous, happening frequently, and not
    a high priority. This is likely used for reading data off of a network card. Interrupts are used
    when you need to handle events that are asynchronous, happening relativly infreaquently, and are of high
    priority. Interrupts are good for real time applications like sensors.

    \item
    What is the function for each bit in the following registers in the ATMega128? EICRA, EICRB, and
    EIMSK. You can find this information from either the AVR Instruction Set guide, or the ATMega128
    Reference Manual both located on the lab web site.  HINT: These registers are related to ‘external
    interrupts.’

    \begin{verbatim}
    Configuration behavior for EICRA and EICRB:
    00 - Low Level triggers an interrupt
    01 - Any logical change
    10 - Falling edge
    11 - Rising edge

    EICRA: (Located a $6A)
    bit 0 => Configuring behavior for interrupt 0
    bit 1 => Configuring behavior for interrupt 0
    bit 2 => Configuring behavior for interrupt 1
    bit 3 => Configuring behavior for interrupt 1
    bit 4 => Configuring behavior for interrupt 2
    bit 5 => Configuring behavior for interrupt 2
    bit 6 => Configuring behavior for interrupt 3
    bit 7 => Configuring behavior for interrupt 3

    EICRB: (Located a $6A)
    bit 0 => Configuring behavior for interrupt 4
    bit 1 => Configuring behavior for interrupt 4
    bit 2 => Configuring behavior for interrupt 5
    bit 3 => Configuring behavior for interrupt 5
    bit 4 => Configuring behavior for interrupt 6
    bit 5 => Configuring behavior for interrupt 6
    bit 6 => Configuring behavior for interrupt 7
    bit 7 => Configuring behavior for interrupt 7

    Configuration behavior for EIMSL:
    0 - Interrupt is disabled
    1 - Interrupr is enabled

    EIMSK: (Located at $39)
    bit 0 => Configuring behavior for interrupt 0
    bit 1 => Configuring behavior for interrupt 1
    bit 2 => Configuring behavior for interrupt 2
    bit 3 => Configuring behavior for interrupt 3
    bit 4 => Configuring behavior for interrupt 4
    bit 5 => Configuring behavior for interrupt 5
    bit 6 => Configuring behavior for interrupt 6
    bit 7 => Configuring behavior for interrupt 7

    \end{verbatim}


    \item
    The AVR microcontroller uses ‘interrupt vectors’ to run code when an interrupt is triggered.
    What is an interrupt vector? List the memory locations for the following vectors in the AVR
    microcontroller: Timer/Counter2 Comparison Match, External Interrupt 2, and USART1-Rx Complete.

    An interrupt vector is a member of the interrupr vector table. The table holds the addresses
    that will called when a specific interrupt is called.

    The memory location of Timer/Counter2 Comparison Match vector is: \$0012
    The memory location of External Interrupt 2 vector is: \$0006
    The memory location of the USART1-Rx Complete Inturrept vector is: \$003C

    \item
    In the AVR microcontroller, like many others, there are several different ways of triggering
    interrupts. Below is a sample signal being input onto one of the external interrupt pins. List where
    the interrupt would trigger on this waveform if the interrupt was set up as: a.) rising edge, b.)
    Falling Edge, c.) Level High, and d.) Level Low.

    \begin{itemize}
        \item
        time 8 and 23
        \item
        time 3 and 20
        \item
        time 9 and 24
        \item
        time 4 and 21
    \end{itemize}

\end{enumerate}
\end{document}
