% template created by: Russell Haering. arr. Joseph Crop
\documentclass[12pt]{article}
\usepackage[hmargin=1in, vmargin=1in]{geometry}
\usepackage{fancyhdr}
\pagestyle{fancy}
\usepackage[hang,small]{caption}
\usepackage{lastpage}
\usepackage{graphicx}
\usepackage{verbatim}
\DeclareGraphicsExtensions{.jpg}
\usepackage{url}

\def\author{Jacques Uber}
\def\title{Lab6 Pre-Lab Questions}
\def\date{\today}

\fancyhf{} % clear all header and footer fields
\fancyhead[LO]{\author}
\fancyhead[RO]{\date}
% The weird spacing here is to get the spacing of \thepage to be right.
\fancyfoot[C]{\thepage\
                    / 7}

                    \setcounter{secnumdepth}{0}
                    \setlength{\parindent}{0pt}
                    \setlength{\parskip}{4mm}
                    \linespread{1.4}


\begin{document}
\fancyhf{} % clear all header and footer fields
\fancyhead[LO]{\author}
\fancyhead[RO]{\date}
\fancyhead[CO]{\title}



\section{Pre-Lab}
\begin{enumerate}
    \item
    In this project you are going to be given a set of functions that you need to have your ‘toy’
    perform. Find another toy that uses a microcontroller and describe the functions that it performs.
    For each function, explain the interesting code (mega128 specific – setup registers ext.),
    mechanics, and/or electronics needed.

    A remote control car would probably have a microcontroller in it. The car would have to listen
    to control signals from the controller. It would have to do what a signal commanded to do it
    when it received a command from the controller. It could listen for a signal and fire an
    interrupt when it received one. This would make the care very responsive to user controls.

    It would definitely need to initialize a stack and enable interrupts, just like the mega128.

\end{enumerate}
\end{document}
